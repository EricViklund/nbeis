\documentclass{beamer}
\mode<presentation>
\usepackage[orientation=portrait,size=a0,scale=1.4]{beamerposter}
\usetheme{gemini}
\usecolortheme{gemini}
\usepackage{siunitx}
\usepackage{wrapfig}
\usepackage{graphicx}
\usepackage{natbib}
 


\usepackage{filecontents}
\begin{filecontents*}{\jobname.bib}
    @article{posen2021advances,
        title={Advances in Nb\textsubscript{3}Sn superconducting radiofrequency cavities towards first practical accelerator applications},
        author={Posen, Sam and Lee, Jaeyel and Seidman, David N and Romanenko, Alexander and Tennis, Brad and Melnychuk, OS and Sergatskov, DA},
        journal={Superconductor Science and Technology},
        volume={34},
        number={2},
        pages={025007},
        year={2021},
        publisher={IOP Publishing}
    }   
\end{filecontents*}

\setbeamertemplate{bibliography entry article}{}
\setbeamertemplate{bibliography entry title}{}
\setbeamertemplate{bibliography entry location}{}
\setbeamertemplate{bibliography entry note}{}


\title{Analysis of Niobium Electropolishing Using Electrochemical Impedance Spectroscopy}%
\author{Eric Viklund \inst{1, 2} \and Tim Ring \inst{2} \and Vijay Chouhan \inst{2} \and David N. Seidman \inst{1} \and Sam Posen \inst{2}}
\institute[shortinst]{\inst{1}Department of Materials Science and Engineering, Northwestern University\\
\inst{2}Fermi National Accelerator Laboratory}
\date{\today}%

\logoleft{\parbox{0.05\textwidth}{\includegraphics[width=7cm]{logos/Northwestern-Logo-White.png} \newline \vspace{1cm} \includegraphics[width=7cm]{logos/FNAL-Logo-White.png}}}
\logoright{\includegraphics[width=7cm]{logos/SRF23-logo.png}}


\begin{document}%
    \begin{frame}{}
        %\maketitle
        \begin{columns}[t]
            \begin{column}{0.32\linewidth}
                \begin{alertblock}{\label{sec:introduction}Key Findings}
                    \begin{itemize}
                        \item Centrifugal barrel polishing (CBP) is an effective method of polishing Nb\textsubscript{3}Sn coated cavities.
                        \item Nb\textsubscript{3}Sn films polished using CBP can acheive less than 20~nm surface roughness with minimal material removal.
                        \item CBP treatment of a Nb\textsubscript{3}Sn cavity leads to a significant increase in the maximum accelerating gradient.
                        \item A secondary recoating procedure is required to achive this increase.
                    \end{itemize}
                \end{alertblock}

            \end{column}





            \begin{column}{0.32\linewidth}    

            \end{column}





            \begin{column}{0.32\textwidth}


                \begin{block}{\label{sec:acknowledgements}Acknowledgements}

                    This manuscript has been authored by Fermi Research Alliance, LLC under Contract No. DE-AC02-07CH11359 with the U.S. Department of Energy, Office of Science, Office of High Energy Physics.

                    \centering \includegraphics[width=10cm]{logos/RGB_Color-Seal_Green-Mark_SC_Horizontal.png}
                \end{block}
            \end{column}
        \end{columns}
    \end{frame}
\end{document}